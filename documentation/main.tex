\documentclass[report]{scrartcl}

% packages
\usepackage{amssymb}
\usepackage{amsmath}
\usepackage{amstext} 
\usepackage{multirow}
\usepackage{listings}
\usepackage{minted}
\usepackage{array}
%\usepackage{parskip}
\usepackage[ngerman]{babel}
\usepackage[dvipsnames]{xcolor}
% Pseudocode
\usepackage{algorithm}
\usepackage{algpseudocode}
\usepackage{graphicx}
\usepackage{gensymb}
\usepackage{mathtools}
\usepackage[colorlinks=true, urlcolor=blue, linkcolor=blue, citecolor=black]{hyperref}
\usepackage{ngerman} % English language
\usepackage[utf8]{inputenc} % Uses the utf8 input encoding
\usepackage[T1]{fontenc} % Use 8-bit encoding that has 256 glyphs
\usepackage[osf]{mathpazo} % Palatino as the main font
\usepackage[backend=biber, style=ieee]{biblatex}
\usepackage{csquotes}
\usepackage{siunitx}
\addbibresource{bibliography.bib}
\linespread{1.15}\selectfont % Palatino needs some extra spacing, here 5% 
\usepackage[
  %showframe,% Seitenlayout anzeigen
  left= 2cm,
  right= 2cm,
  top= 1.8 cm,
  bottom=2.5cm,
  %includeheadfoot
]{geometry}

% enable numbering for paragraphs
\setcounter{secnumdepth}{4}
\setcounter{tocdepth}{4}

\begin{document}


\begin{titlepage}
  \begin{center}
    \includegraphics[]{fbilogo-2.png}
    \vspace*{1cm}

    \Large

    Projekt: Systementwicklung \\
    \textbf{Abschlussbericht} \\

    \vspace{0.5cm}
    \normalsize

    \vspace{7cm}

    \vspace{0.2cm}
    \large
    \textbf{RADACC} \\

    \small{\textbf{R}adar \textbf{A}ided \textbf{D}etection and \textbf{A}nalysis of \textbf{C}rowds in \textbf{C}amera feeds}

    \vspace{1.8cm}

    \normalsize
    Sören Hoock \\
    Leon Mekschrat \\
    Gregor Siano \\
    Matthias Wallenstein \\
    Nikolai Zimmermann

    \vspace{4.cm}

    \small{Betreuung durch}

    Elke Hergenröther \\
    Jens-Peter Akelbein

  \end{center}
\end{titlepage}

%
\thispagestyle{empty}
\newpage
{\hypersetup{linkcolor=black}
  % or \hypersetup{linkcolor=black}, if the colorlinks=true option of hyperref is used
  \tableofcontents
}

\thispagestyle{empty}
\newpage
\setcounter{page}{1}

\part{Projektidee}

% Die ursprüngliche Idee, sowie ihre Entwicklung im Verlauf des Projektes und
% ihre finale Version.

Unsere erste Idee war es mithilfe einer Kamera, die an einem Engpunkt, wie
einem Durchgang oder einer Unterführung, angebracht ist, zu zählen wie viele
Menschen sich an der Stelle aufhalten. Darüber wollten wir Rückschlüsse ziehen
ob es sich um eine kritische Masse handelt bei der eine Massenpanik auftreten
könnte. Mithilfe eines Radar wollten wir die Geschwindigkeit der Masse
bestimmen.

Im Laufe des Projektes haben sich unsere Ziele verändert und schießlich haben
wir uns dazu entschieden die Kamera zu nutzen um die Anzahl an Personen zu
zählen, die eine imaginären Linie überschreiten. Das Radar wollten wir
weiterhin nutzen um die Geschwindigkeit der Personen zu messen.

Die erhobenen Daten sollen in einem Webinterface dargestellt werden und dem
Nutzer die Möglichkeit bieten zu erkennen in welche Bereiche sich viele
Personen aufhalten und ob deren Laufgeschwindigkeit höher ist als normal.

\section{Forschungsfrage}

% Forschungsfrage und ihre Entwicklung im Verlauf des Projekts
% NOTE: Wie wichtig ist diese überhaupt?

Am Anfang des Projektes haben wir eine Forschungsfrage formuliert:

\textit{``Können über die Kombination von Radarsensoren und Kameras die
  Bewegungs-Eigenschaften einer Menschenmenge bestimmt werden?''}

Diese haben wir im Verlauf des Projektes weiter konkretisiert:

\textit{``Kann über die Kombination von Radarsensoren und Kameras eine
  Bereichsanalyse bereitgestellt werden, die automatisch Aussagen über das Risiko
  für eine Massenpanik treffen kann, um damit Sicherheitspersonal zu
  unterstützen?''}

\part{Projektziele}

Das Ziel des Projektes ist es mithilfe eines Doppler-Radars und einer Kamera
Informationen über die Anzahl von Personen und deren Geschwindigkeit zu sammeln
und in einem Webinterface darzustellen.

\part{Umsetzung}

\section{Architektur}

% Detailliertes Architekturbild

% Überblick über die Architektur

\section{Kamera}

% Fokus auf die Kamera im Architekturbild

% Beschreibung der Hardware
% Beschreibung des Algorithmus

\section{Radar}

% Fokus auf das Radar im Architekturbild

Mit dem Radar wollten wir zunächst die Geschwindigkeit der verschiedenen
Personen messen. Im laufe des Projektes haben wir festgestellt, dass das mit
dem Radar was wir nutzen nicht so einfach ist einzelne Personen auszumachen.
Deshalb haben wir uns entschieden die durschnittliche Geschwindigkeit zu
messen.

\subsection{Infineon BGT60TR13C}

Das Radar was uns zur Verfügung gestellt wurde ist ein
\textit{Frequency-Modulated Continuous-Wave} Radar im \qty{60}{\giga\hertz}
Frequenzbereich vom Modell \textit{BGT60TR13C} von Infineon. Diese Radar
verfügt über drei verschiedene Antennen was es ermöglicht den Messbereich in
drei Segmente zu unterteilen. Um akkurat die Geschwindigkeit von mehreren
Personen zu messen benötigen wir weit mehr als drei Segmente. Die
durchschnittliche Geschwindigkeit des Messbereich zu erfassen ist dank des
Dopplereffekts durchaus möglich, weswegen wir uns dafür entschieden haben.

Das Radar selber ist auf einem Shield angebracht. Das Shield verfügt über einen
Microchip, ein paar Status LEDs und einem USB Anschluss, über welchen wir Daten
vom Radar erhalten können.

Um das Radar über Software zu steuern haben wir das Radar Development Kit
genutzt. Dieses beinhaltet auch ein paar Beispiele in Python auf denen wir
letzendlich unser Programm basiert haben.

\subsection{Erster Prototyp in Rust}

Die erste Idee für eine Implementierung war in C oder einer vergleichbaren
Sprache mit dem Radar zu kommunizieren. Der Vorteil im Vergleich zu Python ist
die verbesserte Performanz des resultierenden Programms. Wir haben uns für Rust
entschieden, da Rust mit den C Bibliotheken des RDKs kommunizieren kann,
performant ist und Speichersicherheit verspricht.

Mit dem in Rust entwickelten Prototypen konnten wir das Radar steuern und Daten
empfangen. Als wir jedoch versucht haben die empfangenen Daten zu verarbeiten
haben wir gemerkt wie wenig wir über Signalverarbeitung wissen. Im RDK ist eine
API vorhanden die für uns perfekt ist jedoch haben wir es nicht hinbekommen
diese zum laufen zu bekommen.

Aus Zeitgründen haben wir uns für einen anderen Ansatz entschieden.

\subsection{Finale Version in Python}

Im RDK gibt es Beispiele in Python. Auf einem dieser Beispiele in welchem eine
Range-Doppler Map erstellt wurde haben wir unser Programm basiert.
Um das Hintergrundrauschen der Daten herauszufiltern haben wir eine konstanten
Falschalarmraten Algorithmus genutzt. Von den resultierenden haben wir die
durschnittliche Geschwindigkeit errechnet und diese an den MQTT
Broker geschickt.

\section{Zentraler Server}

\subsection{Konzept}

Unsere initiale Idee des zentralen Servers war ein einfaches Interface das den
aktuellen Status der einzelnen Komponenten widerspiegelt. Dabei musste der
zentrale Server nur die eingehenden Nachrichten empfangen und ausgeben können.
In unserem initialen Design war der Server nicht anderes als ein MQTT Listener
der einfach alle eingehenden Nachrichten ausgegeben hat.

\subsubsection{Nachteile des ersten Design}

Diese erste Designidee war kein wirklicher zentraler Server und mehr ein
sondern nur ein weg die gesendeten Nachrichten abzuhören.

Dies brachte einige Nachteile mit sich unter anderem gab es keinen Kontext zu
Nachrichten. Was bedeutet das abhängig davon wann der das Programm gestartet
wurden andere Informationen zu sehen waren.

Wenn also zwei Personen zeitversetzt das Prskolnik09ogramm starteten waren
unterschiedliche Zustände zu sehen.

Des weiteren mussten die Analyse des Bereichs Zustand direkt in unsere Kamera
und Radar Komponenten erfolgen was bedeutet das diese nicht nur zusätzliche
Informationen wie die Bereichts Koordinaten speichern mussten sonder sich auch
mit anderen Komponenten desselben Bereiches absprechen mussten.

\subsubsection{Überarbeitetes Konzept}

Wir wurden mit der Kamera Komponente um einiges schneller fertig als
ursprünglich geplant was bedeutete das wir viele Ressourcen frei hatten um an
anderen Teilen des Projekts zu arbeiten.

Deswegen und wegen der vorher genannten Probleme mit unserem ersten Konzept
haben wir uns entschieden unseren zentralen Server weiter auszubauen.

Dabei haben wir die Aufgaben des zentralen Server konkreter definiert die
Hauptaufgabe des zentralen Servers soll es sein die erfassten Daten der
einzelnen Kamera und Radar Komponenten zu analysieren und visualisieren. Damit
verlagern wir die analyse der erfassten Daten in eine zentrale Instanz die
einen überblick über das gesamte System hat was es uns erlaubt auch Bereiche
mit mehreren eingängen und Ausgängen zu überwachen.

Der zentralen Server soll außerdem auch alle zusätzlich benötigten
Informationen wie z.B. die maximale Anzahl an Personen in den Bereichen oder
die zugehörigen Koordinaten verwalten. Was bedeutet das die einzelnen
Komponenten nur noch wissen müssen für welchen Bereich sie zuständig sind und
damit keine weiteren Informationen verwalten müssen.

Diese Aufgaben haben wir dann nochmal in Front und Backend unterteilt das
Backend ist in unserem System ausschließlich für Verwaltung der statischen
Raumdaten verantwortlich während dessen das Frontend für das Empfangen,
Analysieren und Darstellen der Daten zuständig ist wir treffen diese
Entscheidung weil, uns die einzelnen Kamera und Radar Komponenten ca. alle 0,2
Sekunden ihren aktuellen Zustand schicken und das in vielen Datenbankzugriffen
enden würde was unser System unter Umständen verlangsamen könnte.

\subsection{Genutzte Technologien}

Die verwendete Programmiersprache ist Typescript und wir verwenden eine
vielzahl an verschiedenen Frameworks die uns das Entwickeln einer Webseite
vereinfachen. Für API Backend verwenden wir Express sowie eine MySql Datenbank.
Als Grundlage des Frontends verwenden wir React. Zum Stylen nutzen Material UI
und für das die Übersichtskarte Leaflet.js.

\subsection{Verwalten der Statischen Bereich Informationen}

Im Backend werden alle statischen Bereich Daten in einer MySql Datenbank
gespeichert die dann über eine Rest API abgefragt werden können. Mit statischen
Bereichs Daten sind alle Informationen über einen Bereich gemeint die sich
nicht ständig ändern oder gar nicht ändern. Ins unserem Falle sind das die
Felder.

\begin{description}
  \item[Id] Eine einzigartige Id die als Primärschlüssel dient.
  \item[Name] Eine vom Kunden frei vergebener Name für den Bereich
  \item[MaxPeople] Die maximale Anzahl an Personen die sich gleichzeitig in einem
    Bereich aufhalten dürfen. (Beachten Sie, dass sich dieser Wert von Zeit zu Zeit
    ändern könnte.)
  \item[Koordinaten] Die Koordinaten sind aufgeteilt in 2 paare Latitude und Longitude
    die jeweils gegenüberliegenden Eckpukte des Bereiches beschreiben
\end{description}

Theoretisch ist es möglich die Bereiche auch über die Rest API zu erstellen und
aktualisieren aber wir haben der einfachheit halber die Daten manuell angelegt.

\subsection{Zuordnung}

Die erste Herausforderung war es die empfangen Nachrichten den dazugehörigen
Bereichen zuzuordnen. Ziel war es dabei

\subsection{Analysieren}

\subsection{Visualisieren}


\section{Simulator}

% Erklärung über den Bedarf
% Erklärung der Simulation

\part{Endergebnis}
\part{Installation}
\part{Nutzung}
\part{Retrospektive}


% ----------------------------
%          Bibliography 
% ----------------------------
\newpage
\nocite{*}
\printbibliography[heading=bibintoc, title={Literatur}]

\end{document}
